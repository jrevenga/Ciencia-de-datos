\documentclass[a4paper, 12pt]{article}
\usepackage[spanish]{babel}
\usepackage[hmargin=2.5cm,vmargin=2.5cm]{geometry}
\usepackage{graphicx}
\usepackage{amsmath}
\usepackage{amssymb}
\usepackage{fancyhdr}

\setlength{\arrayrulewidth}{0.4mm}

\newcommand{\HRule}{\rule{\linewidth}{0.5mm}}

\usepackage{Sweave}
\begin{document}
\Sconcordance{concordance:Grupo.16-PL2.tex:Grupo.16-PL2.Rnw:1 12 1 1 0 104 1 1 2 1 0 %
1 1 14 0 1 2 1 1 1 2 1 0 1 1 8 0 1 2 1 1 1 2 24 0 1 2 1 1 1 2 15 0 1 2 %
1 1 1 2 1 0 2 1 9 0 1 1 10 0 1 2 1 1 1 2 10 0 1 1 10 0 1 2 14 1 1 2 1 0 %
1 1 10 0 2 2 1 0 1 1 12 0 1 2 3 1 1 2 77 0 2 2 99 0 2 2 96 0 1 2 12 1 1 %
2 1 0 1 1 15 0 1 2 1 1 1 2 4 0 1 2 11 1 1 2 11 0 1 2 1 1 1 2 4 0 1 2 1 %
1 1 2 23 0 1 2 1 1 1 2 7 0 2 1 3 0 1 2 19 1 1 3 2 0 1 2 1 1 80 0 1 3 1 %
0 1 3 1 0 2 1 1 3 1 0 2 1 1 4 21 0 1 3 21 0 1 3 11 0 1 3 11 0 1 3 16 0 %
1 4 23 1 1 2 1 0 2 1 20 0 1 1 8 0 1 2 12 1 1 2 1 0 3 1 20 0 1 1 8 0 1 2 %
12 1 1 2 1 0 2 1 20 0 1 1 8 0 1 2 9 1 1 6 5 0 1 8 6 0 1 8 6 0 1 12 10 0 %
1 94 95 0 1 2 2 1 1 2 196 0 2 2 196 0 2 2 196 0 1 2 12 1 1 2 1 0 1 1 16 %
0 2 2 1 0 1 1 3 0 2 2 4 0 2 2 1 0 1 1 3 0 2 2 16 0 1 2 9 1 1 2 1 0 1 1 %
1 3 1 0 3 1 1 3 1 0 3 1 1 3 10 0 1 1 9 0 1 1 9 0 1 1 9 0 1 3 1 0 3 1 1 %
3 6 0 1 1 5 0 1 1 5 0 1 1 6 0 1 2 12 1}

	\begin{titlepage}
		\begin{center}
			\includegraphics[width=0.5\textwidth]{logoUAH.png}~\\[2cm]
			
			\textsc{\Large \\Fundamentos de la Ciencia de Datos}\\[2cm]
			
			\HRule \\[0.4cm]
			{\LARGE \bfseries Prueba de Laboratorio 2 (PL2) \\[0.4cm]}
			\HRule \\[2cm]
			
			\large\textbf{Jorge Revenga Martín de Vidales}\\
			\large\textbf{Ángel Salgado Aldao}\\
			\large\textbf{Adrián García}\\
			\large\textbf{}\\ Grado en Ingeniería Informática \\ Universidad de Alcalá
			
			\vfill
			
			{\large \today}
		\end{center}
	\end{titlepage}
	% Configura los encabezados y pies de página
	\pagestyle{fancy}
	\fancyhf{} % Limpia todos los encabezados y pies de página actuales
	% Encabezado
	\fancyhead[RO,LE]{\textit{Fundamentos de la ciencia de datos}}
	\fancyhead[LO,RE]{\textit{Prueba de Laboratorio 2 (PL2)}}
	% Pie de página
	\fancyfoot[LO,RE]{\textit{Universidad de Alcalá}}
	\fancyfoot[RO,LE]{\thepage}  % Número de página en la esquina inferior derecha
	\newpage
	
	\thispagestyle{plain}
	\tableofcontents
	
	
	\newpage
	
	\section{Introducción - Consideraciones previas}
	
	\subsection{Uso de RStudio}
	Para utilizar una función en R se escribe el nombre de la función, seguido de los parámetros de entrada entre paréntesis e.g.: \texttt{función(parámetros)}
	
	\begin{itemize}
		\item[-]Función \texttt{contributors()}: Muestra los creadores del programa (R)
		
		\item[-]Función \texttt{help()}: Abre un HTML con información sobre la función \texttt{help()} o de la función entre paréntesis de haberla. Para todas las funciones que programemos (para todas las que existan) en R debe poder usarse la función \texttt{help()}.
		
		En el archivo HTML se distinguen varios elementos:
		
		\begin{itemize}
			\item función \{paquete\}: la función de la que se obtiene información seguida del paquete al que pertenece.
			\item Description: descripción de la función.
			\item Usage: aparece la función y todos los argumentos que se le pueden introducir.
			\item Arguments: Explicación de los argumentos o parámetros.
			\item Details: Detalles adicionales de la función.
			\item Offline help: Ayuda sin conexión.
			\item Note: Nota del autor.
			\item References: Referencias.
			\item Examples: Ejemplos de uso de la función.
		\end{itemize}
		
		\item[-]Función \texttt{getwd()} se utiliza para obtener el directorio de trabajo actual (working directory).
		
		\item[-]Función \texttt{setwd(\string"C:/…")} permite cambiar el nuevo directorio de trabajo en el que queramos trabajar.
		
		\item[-] \texttt{help.start()}: Manda a un compendio de todas las ayudas disponibles para trabajar con R.
		
		\item[-]Función \texttt{list.files()}: Muestra todos los archivos en el directorio. \texttt{dir()} hace lo mismo. 
	\end{itemize}
	
	\subsection{Introducción de datos de Excel/CSV en R}
	Para poder leer archivos .xlsx y .csv seguimos un par de pasos:

	\begin{itemize}
		\item[-] Archivos .csv:
		\begin{itemize}
			\item Primero se crea el archivo .csv y se introducen los datos por filas (los datos de la fila i, se escriben en la celda (i,1)) y separando dichos datos con un delimitador ("," o ";").
		\end{itemize}
		
		\item[-] Archivos .xlsx:
		\begin{itemize}
			\item Primero se crea el archivo .xlsc y se introducen los datos en forma de matriz, escribiendo cada dato en una celda.
		\end{itemize}
	\end{itemize}
	
	
  \newpage
	
	\section{Ejercicios con ayuda del profesor}
	Realización de cuatro ejercicios con ayuda del profesor en los que se van a realizar, utilizando el entorno R, dos análisis de clasificación no supervisada y dos análisis de clasificación supervisada, aplicando todos los conceptos teóricos vistos en cada lección.

	\subsection{Análisis de clasificación no supervisada}
	
	\subsubsection{k-Means}
	
	El primer conjunto de datos, que se empleará para realizar el análisis de clasificación no supervisada con k-Means, estará formado por las siguientes 8 calificaciones de estudiantes: 1.\{4, 4\}; 2.\{3, 5\}; 3.\{1, 2\}; 4.\{5, 5\}; 5.\{0, 1\}; 6.\{2, 2\}; 7.\{4, 5\}; 8.\{2, 1\}, donde las características de las calificaciones son: \{Teoría, Laboratorio\}.
	
	\paragraph{Solución:}
	\begin{itemize}
		\item \texttt{m<-matrix(c(4,4, 3,5, 1,2, 5,5, 0,1, 2,2, 4,5, 2,1),2,8)}: 
		Explicacion
\begin{Schunk}
\begin{Sinput}
> m<-matrix(c(4,4, 3,5, 1,2, 5,5, 0,1, 2,2, 4,5, 2,1),2,8)
> (m<-t(m))
\end{Sinput}
\begin{Soutput}
     [,1] [,2]
[1,]    4    4
[2,]    3    5
[3,]    1    2
[4,]    5    5
[5,]    0    1
[6,]    2    2
[7,]    4    5
[8,]    2    1
\end{Soutput}
\end{Schunk}
		\item \texttt{c<-matrix(c(0,1,2,2),2,2)}: 
		Explicacion
\begin{Schunk}
\begin{Sinput}
> c<-matrix(c(0,1,2,2),2,2)
> (c<-t(c))
\end{Sinput}
\begin{Soutput}
     [,1] [,2]
[1,]    0    1
[2,]    2    2
\end{Soutput}
\end{Schunk}
		\item \texttt{(clasificacionns=(kmeans(m,c,4)))}: 
		Explicacion
\begin{Schunk}
\begin{Sinput}
> (clasificacionns=(kmeans(m,c,4)))
\end{Sinput}
\begin{Soutput}
K-means clustering with 2 clusters of sizes 4, 4

Cluster means:
  [,1] [,2]
1 1.25 1.50
2 4.00 4.75

Clustering vector:
[1] 2 2 1 2 1 1 2 1

Within cluster sum of squares by cluster:
[1] 3.75 2.75
 (between_SS / total_SS =  84.8 %)

Available components:

[1] "cluster"      "centers"      "totss"        "withinss"     "tot.withinss"
[6] "betweenss"    "size"         "iter"         "ifault"      
\end{Soutput}
\end{Schunk}
		\item \texttt{}: 
		Explicacion
\begin{Schunk}
\begin{Sinput}
> (m=cbind(clasificacionns$cluster,m))
\end{Sinput}
\begin{Soutput}
     [,1] [,2] [,3]
[1,]    2    4    4
[2,]    2    3    5
[3,]    1    1    2
[4,]    2    5    5
[5,]    1    0    1
[6,]    1    2    2
[7,]    2    4    5
[8,]    1    2    1
\end{Soutput}
\end{Schunk}
		\item \texttt{}: 
		Explicacion
\begin{Schunk}
\begin{Sinput}
> mc1=subset(m,m[,1]==1)
> mc2=subset(m,m[,1]==2)
> mc1
\end{Sinput}
\begin{Soutput}
     [,1] [,2] [,3]
[1,]    1    1    2
[2,]    1    0    1
[3,]    1    2    2
[4,]    1    2    1
\end{Soutput}
\begin{Sinput}
> mc2
\end{Sinput}
\begin{Soutput}
     [,1] [,2] [,3]
[1,]    2    4    4
[2,]    2    3    5
[3,]    2    5    5
[4,]    2    4    5
\end{Soutput}
\end{Schunk}
		\item \texttt{}: 
		Explicacion
\begin{Schunk}
\begin{Sinput}
> (mc1=mc1[,-1])
\end{Sinput}
\begin{Soutput}
     [,1] [,2]
[1,]    1    2
[2,]    0    1
[3,]    2    2
[4,]    2    1
\end{Soutput}
\begin{Sinput}
> (mc2=mc2[,-1])
\end{Sinput}
\begin{Soutput}
     [,1] [,2]
[1,]    4    4
[2,]    3    5
[3,]    5    5
[4,]    4    5
\end{Soutput}
\end{Schunk}
		\item \texttt{}: 
		Explicacion
\begin{Schunk}
\begin{Sinput}
> install.packages("LearnClust")
\end{Sinput}
\begin{Soutput}
package 'LearnClust' successfully unpacked and MD5 sums checked

The downloaded binary packages are in
	C:\Users\Jorge\AppData\Local\Temp\RtmpmiJSmf\downloaded_packages
\end{Soutput}
\begin{Sinput}
> library(LearnClust)
> search()
\end{Sinput}
\begin{Soutput}
 [1] ".GlobalEnv"         "package:LearnClust" "package:magick"    
 [4] "package:arules"     "package:Matrix"     "package:stats"     
 [7] "package:graphics"   "package:grDevices"  "package:utils"     
[10] "package:datasets"   "package:methods"    "Autoloads"         
[13] "package:base"      
\end{Soutput}
\end{Schunk}
		\item \texttt{}: 
		Explicacion
\begin{Schunk}
\begin{Sinput}
> m<-matrix(c(0.89,2.94, 4.36,5.21, 3.75,1.12, 6.25,3.14, 4.1,1.8, 3.9,4.27),2,6)
> (m<-t(m))
\end{Sinput}
\begin{Soutput}
     [,1] [,2]
[1,] 0.89 2.94
[2,] 4.36 5.21
[3,] 3.75 1.12
[4,] 6.25 3.14
[5,] 4.10 1.80
[6,] 3.90 4.27
\end{Soutput}
\end{Schunk}
		\item \texttt{}: 
		Explicacion
\begin{Schunk}
\begin{Sinput}
> agglomerativeHC(m,'EUC','MIN')
\end{Sinput}
\begin{Soutput}
$dendrogram
Number of objects: 6 


$clusters
$clusters[[1]]
    X1   X2
1 0.89 2.94

$clusters[[2]]
    X1   X2
1 4.36 5.21

$clusters[[3]]
    X1   X2
1 3.75 1.12

$clusters[[4]]
    X1   X2
1 6.25 3.14

$clusters[[5]]
   X1  X2
1 4.1 1.8

$clusters[[6]]
   X1   X2
1 3.9 4.27

$clusters[[7]]
    X1   X2
1 3.75 1.12
2 4.10 1.80

$clusters[[8]]
    X1   X2
1 4.36 5.21
2 3.90 4.27

$clusters[[9]]
    X1   X2
1 3.75 1.12
2 4.10 1.80
3 4.36 5.21
4 3.90 4.27

$clusters[[10]]
    X1   X2
1 6.25 3.14
2 3.75 1.12
3 4.10 1.80
4 4.36 5.21
5 3.90 4.27

$clusters[[11]]
    X1   X2
1 0.89 2.94
2 6.25 3.14
3 3.75 1.12
4 4.10 1.80
5 4.36 5.21
6 3.90 4.27


$groupedClusters
  cluster1 cluster2
1        3        5
2        2        6
3        7        8
4        4        9
5        1       10
\end{Soutput}
\end{Schunk}
		\item \texttt{}: 
		Explicacion
\begin{Schunk}
\begin{Sinput}
> agglomerativeHC.details(m,'EUC','MIN')
\end{Sinput}
\begin{Soutput}
[[1]]
     [,1] [,2] [,3]
[1,] 0.89 2.94    1

[[2]]
     [,1] [,2] [,3]
[1,] 4.36 5.21    1

[[3]]
     [,1] [,2] [,3]
[1,] 3.75 1.12    1

[[4]]
     [,1] [,2] [,3]
[1,] 6.25 3.14    1

[[5]]
     [,1] [,2] [,3]
[1,]  4.1  1.8    1

[[6]]
     [,1] [,2] [,3]
[1,]  3.9 4.27    1

         [,1]     [,2]      [,3]     [,4]      [,5]     [,6]
[1,] 0.000000 4.146541 3.3899853 5.363730 3.4064204 3.290745
[2,] 4.146541 0.000000 4.1352388 2.803034 3.4198977 1.046518
[3,] 3.389985 4.135239 0.0000000 3.214094 0.7647876 3.153569
[4,] 5.363730 2.803034 3.2140940 0.000000 2.5333969 2.607566
[5,] 3.406420 3.419898 0.7647876 2.533397 0.0000000 2.478084
[6,] 3.290745 1.046518 3.1535694 2.607566 2.4780839 0.000000
    X1   X2
1 3.75 1.12
2 4.10 1.80
         [,1]     [,2] [,3]     [,4] [,5]     [,6]     [,7]
[1,] 0.000000 4.146541    0 5.363730    0 3.290745 3.389985
[2,] 4.146541 0.000000    0 2.803034    0 1.046518 3.419898
[3,] 0.000000 0.000000    0 0.000000    0 0.000000 0.000000
[4,] 5.363730 2.803034    0 0.000000    0 2.607566 2.533397
[5,] 0.000000 0.000000    0 0.000000    0 0.000000 0.000000
[6,] 3.290745 1.046518    0 2.607566    0 0.000000 2.478084
[7,] 3.389985 3.419898    0 2.533397    0 2.478084 0.000000
    X1   X2
1 4.36 5.21
2 3.90 4.27
         [,1] [,2] [,3]     [,4] [,5] [,6]     [,7]     [,8]
[1,] 0.000000    0    0 5.363730    0    0 3.389985 3.290745
[2,] 0.000000    0    0 0.000000    0    0 0.000000 0.000000
[3,] 0.000000    0    0 0.000000    0    0 0.000000 0.000000
[4,] 5.363730    0    0 0.000000    0    0 2.533397 2.607566
[5,] 0.000000    0    0 0.000000    0    0 0.000000 0.000000
[6,] 0.000000    0    0 0.000000    0    0 0.000000 0.000000
[7,] 3.389985    0    0 2.533397    0    0 0.000000 2.478084
[8,] 3.290745    0    0 2.607566    0    0 2.478084 0.000000
    X1   X2
1 3.75 1.12
2 4.10 1.80
3 4.36 5.21
4 3.90 4.27
          [,1] [,2] [,3]     [,4] [,5] [,6] [,7] [,8]     [,9]
 [1,] 0.000000    0    0 5.363730    0    0    0    0 3.290745
 [2,] 0.000000    0    0 0.000000    0    0    0    0 0.000000
 [3,] 0.000000    0    0 0.000000    0    0    0    0 0.000000
 [4,] 5.363730    0    0 0.000000    0    0    0    0 2.533397
 [5,] 0.000000    0    0 0.000000    0    0    0    0 0.000000
 [6,] 0.000000    0    0 0.000000    0    0    0    0 0.000000
 [7,] 0.000000    0    0 0.000000    0    0    0    0 0.000000
 [8,] 0.000000    0    0 0.000000    0    0    0    0 0.000000
 [9,] 3.290745    0    0 2.533397    0    0    0    0 0.000000
    X1   X2
1 6.25 3.14
2 3.75 1.12
3 4.10 1.80
4 4.36 5.21
5 3.90 4.27
          [,1] [,2] [,3] [,4] [,5] [,6] [,7] [,8] [,9]    [,10]
 [1,] 0.000000    0    0    0    0    0    0    0    0 3.290745
 [2,] 0.000000    0    0    0    0    0    0    0    0 0.000000
 [3,] 0.000000    0    0    0    0    0    0    0    0 0.000000
 [4,] 0.000000    0    0    0    0    0    0    0    0 0.000000
 [5,] 0.000000    0    0    0    0    0    0    0    0 0.000000
 [6,] 0.000000    0    0    0    0    0    0    0    0 0.000000
 [7,] 0.000000    0    0    0    0    0    0    0    0 0.000000
 [8,] 0.000000    0    0    0    0    0    0    0    0 0.000000
 [9,] 0.000000    0    0    0    0    0    0    0    0 0.000000
[10,] 3.290745    0    0    0    0    0    0    0    0 0.000000
    X1   X2
1 0.89 2.94
2 6.25 3.14
3 3.75 1.12
4 4.10 1.80
5 4.36 5.21
6 3.90 4.27
\end{Soutput}
\end{Schunk}
		\item \texttt{}: 
		Explicacion
\begin{Schunk}
\begin{Sinput}
> agglomerativeHC.details(m,'EUC','MAX')
\end{Sinput}
\begin{Soutput}
[[1]]
     [,1] [,2] [,3]
[1,] 0.89 2.94    1

[[2]]
     [,1] [,2] [,3]
[1,] 4.36 5.21    1

[[3]]
     [,1] [,2] [,3]
[1,] 3.75 1.12    1

[[4]]
     [,1] [,2] [,3]
[1,] 6.25 3.14    1

[[5]]
     [,1] [,2] [,3]
[1,]  4.1  1.8    1

[[6]]
     [,1] [,2] [,3]
[1,]  3.9 4.27    1

         [,1]     [,2]      [,3]     [,4]      [,5]     [,6]
[1,] 0.000000 4.146541 3.3899853 5.363730 3.4064204 3.290745
[2,] 4.146541 0.000000 4.1352388 2.803034 3.4198977 1.046518
[3,] 3.389985 4.135239 0.0000000 3.214094 0.7647876 3.153569
[4,] 5.363730 2.803034 3.2140940 0.000000 2.5333969 2.607566
[5,] 3.406420 3.419898 0.7647876 2.533397 0.0000000 2.478084
[6,] 3.290745 1.046518 3.1535694 2.607566 2.4780839 0.000000
    X1   X2
1 3.75 1.12
2 4.10 1.80
         [,1]     [,2] [,3]     [,4] [,5]     [,6]     [,7]
[1,] 0.000000 4.146541    0 5.363730    0 3.290745 3.406420
[2,] 4.146541 0.000000    0 2.803034    0 1.046518 4.135239
[3,] 0.000000 0.000000    0 0.000000    0 0.000000 0.000000
[4,] 5.363730 2.803034    0 0.000000    0 2.607566 3.214094
[5,] 0.000000 0.000000    0 0.000000    0 0.000000 0.000000
[6,] 3.290745 1.046518    0 2.607566    0 0.000000 3.153569
[7,] 3.406420 4.135239    0 3.214094    0 3.153569 0.000000
    X1   X2
1 4.36 5.21
2 3.90 4.27
         [,1] [,2] [,3]     [,4] [,5] [,6]     [,7]     [,8]
[1,] 0.000000    0    0 5.363730    0    0 3.406420 4.146541
[2,] 0.000000    0    0 0.000000    0    0 0.000000 0.000000
[3,] 0.000000    0    0 0.000000    0    0 0.000000 0.000000
[4,] 5.363730    0    0 0.000000    0    0 3.214094 2.803034
[5,] 0.000000    0    0 0.000000    0    0 0.000000 0.000000
[6,] 0.000000    0    0 0.000000    0    0 0.000000 0.000000
[7,] 3.406420    0    0 3.214094    0    0 0.000000 4.135239
[8,] 4.146541    0    0 2.803034    0    0 4.135239 0.000000
    X1   X2
1 6.25 3.14
2 4.36 5.21
3 3.90 4.27
         [,1] [,2] [,3] [,4] [,5] [,6]     [,7] [,8]     [,9]
 [1,] 0.00000    0    0    0    0    0 3.406420    0 5.363730
 [2,] 0.00000    0    0    0    0    0 0.000000    0 0.000000
 [3,] 0.00000    0    0    0    0    0 0.000000    0 0.000000
 [4,] 0.00000    0    0    0    0    0 0.000000    0 0.000000
 [5,] 0.00000    0    0    0    0    0 0.000000    0 0.000000
 [6,] 0.00000    0    0    0    0    0 0.000000    0 0.000000
 [7,] 3.40642    0    0    0    0    0 0.000000    0 4.135239
 [8,] 0.00000    0    0    0    0    0 0.000000    0 0.000000
 [9,] 5.36373    0    0    0    0    0 4.135239    0 0.000000
    X1   X2
1 0.89 2.94
2 3.75 1.12
3 4.10 1.80
      [,1] [,2] [,3] [,4] [,5] [,6] [,7] [,8]    [,9]   [,10]
 [1,]    0    0    0    0    0    0    0    0 0.00000 0.00000
 [2,]    0    0    0    0    0    0    0    0 0.00000 0.00000
 [3,]    0    0    0    0    0    0    0    0 0.00000 0.00000
 [4,]    0    0    0    0    0    0    0    0 0.00000 0.00000
 [5,]    0    0    0    0    0    0    0    0 0.00000 0.00000
 [6,]    0    0    0    0    0    0    0    0 0.00000 0.00000
 [7,]    0    0    0    0    0    0    0    0 0.00000 0.00000
 [8,]    0    0    0    0    0    0    0    0 0.00000 0.00000
 [9,]    0    0    0    0    0    0    0    0 0.00000 5.36373
[10,]    0    0    0    0    0    0    0    0 5.36373 0.00000
    X1   X2
1 6.25 3.14
2 4.36 5.21
3 3.90 4.27
4 0.89 2.94
5 3.75 1.12
6 4.10 1.80
\end{Soutput}
\end{Schunk}
		\item \texttt{}: 
		Explicacion
\begin{Schunk}
\begin{Sinput}
> 
\end{Sinput}
\end{Schunk}

		\end{itemize}
	
	
	\subsubsection{Clusterización Jerárquica Aglomerativa}
	
	El segundo conjunto de datos, que se empleará para realizar el análisis de clasificación no supervisada con Clusterización Jerárquica Aglomerativa, estará formado por 6 calificaciones de estudiantes: 1.\{0.89, 2.94\}; 2.\{4.36, 5.21\}; 3.\{3.75, 1.12\}; 4.\{6.25, 3.14\}; 5.\{4.1, 1.8\}; 6.\{3.9, 4.27\}.
	
	\paragraph{Solución}
	
	\newpage

	\subsection{Análisis de clasificación supervisada}
	
	\subsubsection{Árboles de decisión}
	
	El tercer conjunto de datos, que se empleará para realizar el análisis de clasificación supervisada utilizando árboles de decisión, estará formado por las siguientes 9 calificaciones de estudiantes: 1.\{A,A,B,Ap\}; 2.\{A,B,D,Ss\}; 3.\{D,D,C,Ss\}; 4.\{D,D,A,Ss\}; 5.\{B,C,B,Ss\}; 6.\{C,B,B,Ap\}; 7.\{B,B,A,Ap\}; 8.\{C,D,C,Ss\}; 9.\{B,A,C,Ss\}, donde las características de las calificaciones son: \{Teoría, Laboratorio, Prácticas, Calificación Global).
	
	
	\subsubsection{Regresión}
	
	El cuarto conjunto de datos, que se empleará para realizar el análisis de clasificación supervisada utilizando regresión, estará formado por los siguientes 4 radios ecuatoriales y densidades de los planetas interiores: \{Mercurio,2.4,5.4; Venus,6.1,5.2; Tierra,6.4,5.5; Marte,3.4,3.9\}
	
	
	\newpage
	
	\section{Ejercicios de forma autónoma}
	
	Realización de cuatro ejercicios de forma autónoma por cada grupo de estudiantes en los que se van a realizar, utilizando el entorno R, dos análisis de clasificación no supervisada y dos análisis de clasificación supervisada, aplicando todos los conceptos teóricos vistos en cada lección.
	
	\subsection{Análisis de clasificación no supervisada}
	
	\subsubsection{K-means}
	
	El primer conjunto de datos, que se empleará para realizar el análisis de clasificación no supervisada con K-means, estará formado por los siguientes 15 valores de velocidades de respuesta y temperaturas normalizadas de un microprocesador \{Velocidad, Temperatura\}: 1.\{3.5, 4.5\}; 2.\{0.75, 3.25\}; 3.\{0, 3\}; 4.\{1.75, 0.75\}; 5.\{3, 3.75\}; 6.\{3.75, 4.5\}; 7.\{1.25, 0.75\}; 8.\{0.25, 3\}; 9.\{3.5, 4.25\}; 10.\{1.5, 0.5\}; 11.\{1, 1\}; 12.\{3, 4\}; 13.\{0.5, 3\}; 14.\{2, 0.25\}; 15.\{0, 2.5\}. Del análisis visual de los datos se ha concluido que hay una alta probabilidad que sean tres clusters.
	
	\paragraph{Solución:}
	
	
	\subsubsection{Clusterización Jerárquica Aglomerativa}
	
	El segundo conjunto de datos, que se empleará para realizar el análisis de clasificación no supervisada con Clusterización Jerárquica Aglomerativa, será el mismo que el utilizado en el ejercicio anterior, por lo tanto estará formado por los siguientes 15 valores de velocidades de respuesta y temperaturas normalizadas de un microprocesador \{Velocidad, Temperatura\}: 1.\{3.5, 4.5\}; 2.\{0.75, 3.25\}; 3.\{0, 3\}; 4.\{1.75, 0.75\}; 5.\{3, 3.75\}; 6.\{3.75, 4.5\}; 7.\{1.25, 0.75\}; 8.\{0.25, 3\}; 9.\{3.5, 4.25\}; 10.\{1.5, 0.5\}; 11.\{1, 1\}; 12.\{3, 4\}; 13.\{0.5, 3\}; 14.\{2, 0.25\}; 15.\{0, 2.5\}. Del análisis visual de los datos se ha concluido que hay una alta probabilidad que sean tres clusters.
	
	\paragraph{Solución:}
	\begin{itemize}
		\item \texttt{datos <- read.csv("Datos2.2.csv")}: Se leen los datos del archivo .csv y se guardan en la variable "datos".
		\item \texttt{(datos<-t(datos))}: Se hace la traspuesta de la matriz datos para un mejor tratamiento.
		\begin{itemize}
			\item [-] Parámetros:
			\begin{itemize}
				\item \texttt{datos}: Variable que contiene los puntos a evaluar.
				\item \texttt{"single"}: Algoritmo de clusterización, en este caso usando minimum linkage.
				\item \texttt{details="FALSE"}: Booleano para determinar si se imprimen o no los logs con la explicación de todo el proceso.
				\item \texttt{waiting="FALSE"}: Booleano para determinar si tiene que esperar input del usuario para ir imprimiendo los logs.
			\end{itemize}
			\item [-] Salida: Matriz con los clusters y sus distancias.
\begin{Schunk}
\begin{Sinput}
> library("clustlearn")
> datos <- read.csv("Datos2.2.csv")
> (datos<-t(datos))
\end{Sinput}
\begin{Soutput}
       [,1]
X3.5   4.50
X0.75  3.25
X0     3.00
X1.75  0.75
X3     3.75
X3.75  4.50
X1.25  0.75
X0.25  3.00
X3.5.1 4.25
X1.5   0.50
X1     1.00
X3.1   4.00
X0.5   3.00
X2     0.25
X0.1   2.50
\end{Soutput}
\begin{Sinput}
> (clMin<-agglomerative_clustering(datos,"single",FALSE,FALSE))
\end{Sinput}
\begin{Soutput}
Cluster method   : single 
Distance         : Euclidean 
Number of objects: 15 
\end{Soutput}
\end{Schunk}
		\end{itemize}

		\begin{itemize}
			\item [-] Parámetros:
			\begin{itemize}
				\item \texttt{datos}: Variable que contiene los puntos a evaluar.
				\item \texttt{"single"}: Algoritmo de clusterización, en este caso usando maximum linkage.
				\item \texttt{details="FALSE"}: Booleano para determinar si se imprimen o no los logs con la explicación de todo el proceso.
				\item \texttt{waiting="FALSE"}: Booleano para determinar si tiene que esperar input del usuario para ir imprimiendo los logs.
			\end{itemize}
			\item [-] Salida: Matriz con los clusters y sus distancias.
\begin{Schunk}
\begin{Sinput}
> install.packages("clustlearn")
> library("clustlearn")
> datos <- read.csv("Datos2.2.csv")
> (datos<-t(datos))
\end{Sinput}
\begin{Soutput}
       [,1]
X3.5   4.50
X0.75  3.25
X0     3.00
X1.75  0.75
X3     3.75
X3.75  4.50
X1.25  0.75
X0.25  3.00
X3.5.1 4.25
X1.5   0.50
X1     1.00
X3.1   4.00
X0.5   3.00
X2     0.25
X0.1   2.50
\end{Soutput}
\begin{Sinput}
> (clMin<-agglomerative_clustering(datos,"complete",FALSE,FALSE))
\end{Sinput}
\begin{Soutput}
Cluster method   : complete 
Distance         : Euclidean 
Number of objects: 15 
\end{Soutput}
\end{Schunk}
		\end{itemize}

		\begin{itemize}
			\item [-] Parámetros:
			\begin{itemize}
				\item \texttt{datos}: Variable que contiene los puntos a evaluar.
				\item \texttt{"average"}: Algoritmo de clusterización, en este caso usando average linkage.
				\item \texttt{details="FALSE"}: Booleano para determinar si se imprimen o no los logs con la explicación de todo el proceso.
				\item \texttt{waiting="FALSE"}: Booleano para determinar si tiene que esperar input del usuario para ir imprimiendo los logs.
			\end{itemize}
			\item [-] Salida: Matriz con los clusters y sus distancias.
\begin{Schunk}
\begin{Sinput}
> library("clustlearn")
> datos <- read.csv("Datos2.2.csv")
> (datos<-t(datos))
\end{Sinput}
\begin{Soutput}
       [,1]
X3.5   4.50
X0.75  3.25
X0     3.00
X1.75  0.75
X3     3.75
X3.75  4.50
X1.25  0.75
X0.25  3.00
X3.5.1 4.25
X1.5   0.50
X1     1.00
X3.1   4.00
X0.5   3.00
X2     0.25
X0.1   2.50
\end{Soutput}
\begin{Sinput}
> (clMin<-agglomerative_clustering(datos,"average",FALSE,FALSE))
\end{Sinput}
\begin{Soutput}
Cluster method   : average 
Distance         : Euclidean 
Number of objects: 15 
\end{Soutput}
\end{Schunk}
		\end{itemize}
	\end{itemize}
	
	\subsection{Análisis de clasificación supervisada}
	
	\subsubsection{Árboles de decisión}
	
	El tercer conjunto de datos, que se empleará para realizar el análisis de clasificación supervisada utilizando árboles de decisión, estará formado por el siguiente conjunto de 10 sucesos constituidos por los valores de cuatro características de vehículos: 1.\{B,4,5,Coche\}; 2.\{A,2,2,Moto\}; 3.\{N,2,1,Bicicleta\}; 4.\{B,6,4,Camión\}; 5.\{B,4,6,Coche\}; 6.\{B,4,4,Coche\}; 7.\{N,2,2,Bicicleta\}; 8.\{B,2,1,Moto\}; 9.\{B,6,2,Camión\}; 10.\{N,2,1,Bicicleta\}, donde las características de cada suceso son: \{TipoCarnet, NúmeroRuedas, NúmeroPasajeros, TipoVehículo\}. Se debe clasificar el tipo de vehículo en función del resto de características. TipoCarnet, es el tipo de carnet necesario para conducir el vehículo.
	
	
	\subsubsection{Regresión}
	
	El cuarto conjunto de datos, que se empleará para realizar el análisis de clasificación supervisada utilizando regresión, estará formado por los siguientes 4 subconjuntos de datos: 1.\{10, 8.04; 8, 6.95; 13, 7.58; 9, 8.81; 11, 8.33; 14, 9.96; 6, 7.24; 4, 4.26; 12, 10.84; 7, 4.82; 5, 5.68\}; 2.\{10, 9.14; 8, 8.14; 13, 8.74; 9, 8.77; 11, 9.26; 14, 8.1; 6, 6.13; 4, 3.1; 12, 9.13; 7, 7.26; 5, 4.74\}; 3.\{10, 7.46; 8, 6.77; 13, 12.74; 9, 7.11; 11, 7.81; 14, 8.84; 6, 6.08; 4, 5.39; 12, 8.15; 7, 6.42; 5, 5.73\}; 4.\{8, 6.58; 8, 5.76; 8, 7.71; 8, 8.84; 8, 8.47; 8, 7.04; 8, 5.25; 19, 12.5; 8, 5.56; 8, 7.91; 8, 6.89\}. Se deben calcular las rectas de regresión de los cuatro subconjuntos y sus parámetros de ajuste.
		
	\paragraph{Solución:}
			Algoritmo de minería de reglas de asociación (apriori): Su objetivo principal es descubrir patrones de asociación entre diferentes conjuntos de datos.
			\begin{itemize}
				\item[-] Parámetros:
				\begin{itemize}
					\item \texttt{transacciones}: Lista de sucesos que conforman la muestra.
					\item \texttt{soporte}: Umbral mínimo de soporte que deben superar las asociaciaciones.
					\item \texttt{confianza}: Umbral mínimo de confianza que deben superar las asociaciaciones.
				\end{itemize}
				
				\item[-] Retorno: 
				
				\item[-] Explicación: 

			\end{itemize}
\begin{Schunk}
\begin{Sinput}
> 
\end{Sinput}
\end{Schunk}

  
\end{document}
