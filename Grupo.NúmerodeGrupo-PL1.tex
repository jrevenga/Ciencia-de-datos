\documentclass[]{article}
\usepackage[spanish]{babel}
%opening
\title{Fundamentos de la Ciencia de Datos. Prueba de Laboratorio 1 (PL1)}
\author{Jorge Revenga}

\begin{document}
	
\maketitle
	
\begin{abstract}
Un poco de texto de resumen
\end{abstract}

\section{Introducción}

Un poco de texto sobre las funciones introductorias Contributors(), Help()...
	
\section{Ejercicios con ayuda del profesor}
Realización de cuatro ejercicios con ayuda del profesor en los que se van a realizar, utilizando el entorno R, un análisis de descripción de datos, un análisis de asociación y dos análisis de detección de datos anómalos, aplicando todos los conceptos teóricos vistos en cada lección.

\subsection{Análisis de descripción de datos}
El primer conjunto de datos, que se empleará para realizar el análisis de descripción de datos, estará formado por datos de una característica cualitativa, nombre, y otra cuantitativa, radio, de los satélites menores de Urano, es decir, aquellos que tienen un radio menor de 50 Km, dichos datos, los primeros cualitativos nominales, y los segundos cuantitativos continuos, son: (Nombre, radio en Km): Cordelia, 13; Ofelia, 16; Bianca, 22; Crésida, 33; Desdémona, 29; Julieta, 42; Rosalinda, 27; Belinda, 34; Luna-1986U10, 20; Calíbano, 30; Luna-999U1, 20; Luna 1999U2, 15.

\subsection{Análisis de asociación}
El segundo conjunto de datos, que se empleará para realizar el análisis de asociación, estará formado por las siguientes 6 cestas de la compra: {Pan, Agua, Leche, Naranjas}, {Pan, Agua, Café, Leche}, {Pan, Agua, Leche}, {Pan, Café, Leche}, {Pan, Agua}, {Leche}.

\subsection{Análisis de detección de datos anómalos}
Esto es un poco más de texto en el párrafo.

\subsubsection{Técnicas con base estadística}
El tercer conjunto de datos, que se empleará para realizar el análisis de detección de datos anómalos utilizando técnicas con base estadística, estará formado por los siguientes 7 valores de resistencia y densidad para diferentes tipos de hormigón {Resistencia, Densidad}: {3, 2; 3.5, 12; 4.7, 4.1; 5.2, 4.9; 7.1, 6.1; 6.2, 5.2; 14, 5.3}. Aplicar las medidas de ordenación a la resistencia y las de dispersión a la densidad.

\subsubsection{Técnicas basadas en la proximidad y en la densidad}
El cuarto conjunto de datos, que se empleará para realizar el análisis de detección de datos anómalos utilizando técnicas basadas en la proximidad y en la densidad, estará formado por las siguientes 5 calificaciones de estudiantes: 1. {4, 4}; 2. {4, 3}; 3. {5, 5}; 4. {1, 1}; 5. {5, 4} donde las características de las calificaciones son: (Teoría, Laboratorio).

\section{Ejercicios de forma autónoma}
Realización de cuatro ejercicios de forma autónoma por cada grupo de estudiantes en los que se van a realizar, utilizando el entorno R, un análisis de descripción de datos, un análisis de asociación y dos análisis de detección de datos anómalos, aplicando todos los conceptos teóricos vistos en cada lección:

\subsection{Análisis de descripción de datos}
El primer conjunto de datos, que se empleará para realizar el análisis de descripción de datos, estará formado por datos de una característica cuantitativa, distancia, desde el domicilio de cada estudiantes hasta la Universidad, dichos datos, cuantitativos continuos, son: 16.5, 34.8, 20.7, 6.2, 4.4, 3.4, 24, 24, 32, 30, 33, 27, 15, 9.4, 2.1, 34, 24, 12, 4.4, 28, 31.4, 21.6, 3.1, 4.5, 5.1, 4, 3.2, 25, 4.5, 20, 34, 12, 12, 12, 12, 5, 19, 30, 5.5, 38, 25, 3.7, 9, 30, 13, 30, 30, 26, 30, 30, 1, 26, 22, 10, 9.7, 11, 24.1, 33, 17.2, 27, 24, 27, 21, 28, 30, 4, 46, 29, 3.7, 2.7, 8.1, 19, 16.

\subsection{Análisis de asociación}
El segundo conjunto de datos, que se empleará para realizar el análisis de asociación, estará formado por las siguientes conjuntos de extras incluidos en 8 ventas de coches: {X, C, N, B}, {X, T, B, C), {N, C, X}, {N, T, X, B}, {X, C, B}, {N}, {X, B, C}, {T, A}. Donde: {X: Faros de Xenon, A: Alarma, T: Techo Solar, N: Navegador, B: Bluetooth, C: Control de Velocidad}, son los extras que se pueden incluir en cada coche
		
\subsection{Análisis de detección de datos anómalos}
Esto es un poco más de texto en el párrafo.

\subsubsection{Técnicas con base estadística}
El tercer conjunto de datos, que se empleará para realizar el análisis de detección de datos anómalos utilizando técnicas con base estadística, estará formado por los siguientes 10 valores de velocidades de respuesta y temperaturas normalizadas de un microprocesador {Velocidad, Temperatura}: {10, 7.46; 8, 6.77; 13, 12.74; 9, 7.11; 11, 7.81; 14, 8.84; 6, 6.08; 4, 5.39; 12, 8.15; 7, 6.42; 5, 5.73}. Aplicar las medidas de ordenación a la velocidad y las de dispersión a la temperatura.

\subsubsection{Técnicas basadas en la proximidad y en la densidad}
El cuarto conjunto de datos, que se empleará para realizar el análisis de detección de datos anómalos utilizando técnicas basadas en la proximidad y en la densidad, estará formado por el número de Mujeres y Hombres inscritos en una serie de cinco seminarios que se han impartido sobre biología. Los datos son: {Mujeres, Hombres}: 1. {9, 9}; 2. {9, 7}; 3. {11, 11}; 4. {2, 1}; 5. {11, 9}.

\end{document}